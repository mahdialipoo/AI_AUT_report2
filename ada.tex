\subsection{همیشه پویا}
یکی از مهمترین مواردی که باید رد نظر گرفته شود محیط های پویاست . از آنجایی که فرد می‌خواهد کل زمان سفر به حداقل برسد، باید برنامه‌ریزی راه‌حل‌ها را با بیشترین سرعتی که می‌تواند محاسبه کند. در صورت برخورد با یک مانع جدید، زمان سفر با زمان اضافی برای محاسبه یک مسیر جدید افزایش می یابد. برای مقابله با این مشکل، الگوریتم
\lr{ ADA*} 
با ترکیب ایده‌های دو الگوریتم قبلاً مورد بحث،
\lr{ D*lite} 
و 
\lr{ARA*}
توسعه یافت. مشابه 
\lr{ARA*}
،
\lr{ AD*} 
راه حل هایی غیر بهینه را ابتدا جست و جو می کند . تغییری که در محیط شناسایی شود (از طریق سنسور یا دوربین)، به عنوان تغییر در گرافی که مسیر در آن برنامه ریزی شده است، منعکس خواهد شد. گره‌های آسیب‌دیده در فهرست باز مانند الگوریتم
\lr{ D* Lite} 
ارسال می‌شوند، با اولویت‌های آنها در میان مقدار کلید قبلی و مقدار کلید به‌روزرسانی می شوند . سپس گره های موجود در لیست پردازش می شوند تا زمانی که راه حل فعلی تضمین شود بنابراین
\lr{ ADA*}
، که یک الگوریتم  برخط است، امکان برنامه ریزی مجدد سریع راه حل ها را با محدود کردن بهینه بودن آنها فراهم می کند. 
