\chapter{مقدمه ای بر الگوریتم \lr{A*}}
\section{تاریخچه}
پیتر هارت 
\lr{(Peter Hart)}
، نیلز نیلسون 
\lr{(Nils Nilsson) }
و برترام رافائل 
\lr{(Bertram Raphael)} 
از موسسه پژوهشی استنفورد 
\lr{(Stanford Research Institute) }
که اکنون با عنوان اس‌آرآی اینترنشنال 
\footnote{\lr{SRI International}} 
فعالیت می‌کند، برای اولین‌بار، مقاله‌ای پیرامون الگوریتم
\lr{ *A }
 را در سال ۱۹۶۳ منتشر کردند. این الگوریتم را می‌توان به عنوان افزونه‌ای از «الگوریتم دیکسترا» 
\footnote{\lr{ Dijkstra's Algorithm}}
  در نظر گرفت که توسط «ادسخر دیکسترا»
\footnote{\lr{  Edsger Dijkstra}}
   در سال ۱۹۵۹ ارائه شده است. الگوریتم 
\lr{   *A}
   با بهره‌گیری از «الگوریتم جستجوی کاشف» (جستجوی هیوریستیک  
   \lr{Heuristics Search}) 
   برای هدایت فرایند جستجو، به کارایی بهتری دست پیدا می‌کند
\cite{ELhamalgoritma_star}
.
\section{حمل نقل و مسیریابی با  الگوریتم \lr{A*}}
توسعه سیستم‌های اتومات مانند هواپیماهای بدون سرنشین، وسایل نقلیه هدایت‌شونده خودکار و ربات‌های خودکار مزایای بسیاری را برای انسان داشته اند . توسعه وسایل نقلیه خودران منجر به افزایش ایمنی جاده ها و بهبود مصرف انرژی شده است. برای خودران سازی وسایل نقلیه باید نوعی سیستم داشت تا مسیرهای خود را مطابق با محیطی که قرار است در آن حرکت کنند برنامه ریزی کند. خواسته ی ما در این گونه مسائل این است که  این مسیرها تا حد امکان کوتاه باشند و وسیله نقلیه به راحتی حرکت کند و از همه مهمتر اینکه بدون مانع باشند .
با این حال، تحقیق در مورد برنامه ریزی حرکتی سیستم های خودران جدید نیست و به دهه 1950 برمی گردد، با الگوریتم هایی مانند جستجوی عرضی و جستجوی عمقی در مرحله اولیه تحقیقات برنامه ریزی حرکتی فرموله شده است. از آن زمان تاکنون چندین پیشرفت بزرگ در توسعه الگوریتم‌های برنامه‌ریزی حرکت صورت گرفته است . 
\cite{paliwal2023survey}
. یکی از الگوریتم های مهم برای هوشمندسازی و توانمد سازی این وسایل برای مسیریابی الگوریتم جست و جوی 
\lr{ََََA*}
می باشد .