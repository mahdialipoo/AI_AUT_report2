\chapter{جمع بندی و نتیجه گیری}
با پیشرفت فناوری های جدید ، ساخت وسایل هوشمند را به همراه داشته است . الگوریتم های یادگیری ماشین شامل یادگیری تقویتی و شبکه های عصبی و همین طور الگوریتم های مربوط به بینایی ماشین در پیشبرد هوشمند سازی وسایل ساخته ی بشر نقش اساسی داشته اند . این عامل های هوشمند هر یک می توانند در مسائل مختلف بر اساس نوعی عقلانیت تصمیم گیری کنند و بدون نیاز به نوشتن الگوریتم های متعدد کار ها و عملیات های متفاوتی را انجام دهند . در این گزارش سعی شد تا به تنوع این عوامل هوش مصنوعی و ویژگی های هریک و محیط هایی که در آن عمل می کنند ، پرداخته  شود .  