
\subsection{پویا ساده}

الگوریتم
\lr{ D* Lite}
،مانند الگوریتم 
\lr{A* }
کلاسیک با شروع از یک گره سعی می کند تابه نقطه ی هدف برسد اما با این فرض که گراف بتواند تغییر کند . تشخیص یک مانع همان اثری را بر روی نمودار خواهد داشت که با حذف یال بین دو گره که مانع بین آنها تشخیص داده می شود، و این معادل تغییر وزن یال به ∞ است. الگوریتم
\lr{ D* Lite}
چنین تغییرات وزنی را در نمودار محاسبه می کند و به طور مداوم مسیر خودروی خودران را دوباره برنامه ریزی می کند.
الگوریتم
\lr{D* Lite}
دو امتیاز را برای هر گره از گراف ذر نظر می گیرد  
\lr{G}
و  
\lr{RHS}
.  که همانند 
\lr{  A* }
امتیاز
\lr{G}
هزینه رفتن از گره شروع به گره فعلی در گراف می باشد. امتیاز

\lr{RHS }
به صورت
$RHS(n) = min(G(n′) + C(n′, n)) $
تعریف می شود که در آن،
\lr{n}
گره فعلی، 
\lr{n′} 
گره قبلی است.  و  
\lr{ C(n′, n)}
هزینه جابجایی از 
\lr{n }
به
\lr{n}
. برای یافتن گره بهینه بعدی استفاده می شود. اگر گره بعدی مسدود شود،
\lr{ C} 
برای آن لبه روی 
$\infty$ 
تنظیم می شود، در نتیجه،
\lr{ RHS}
نیز روی 
$\infty$
تنظیم می شود. با استفاده از این امتیازها، الگوریتم مسیری را از هدف تا شروع برنامه‌ریزی می‌کند که در نهایت یک مسیر بهینه را برای دنبال کردن به دست می‌دهد. در صورت مواجهه با یک مانع ناشناخته قبلی، امتیاز 
\lr{RHS }
با هزینه های جدید اتصال برای گره های آسیب دیده، دوباره محاسبه می شود. نتایج نشان می‌دهد که
\lr{D* Lite }
منجر به جست و جوی راس اضافی کمتر، گسترده شدن کمتر ی می‌شود، که ثابت می‌کند که
\lr{D* Lite}
در واقع الگوریتم خوبی برای برنامه‌ریزی مجدد و حرکت در محیط‌های ناشناخته است.


